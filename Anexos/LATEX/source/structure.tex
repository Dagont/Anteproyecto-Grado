%%	febrero 2018
%%	Autor: Imer A. Robles Rodriguez - imer.robles@correo.uis.edu.co
%%
%%	Archivo que contiene la mayoria de la configuracion complementaria
%%	que no pudo ser agregada en la clase 'icontecUIS.cls'
%%	incluirlo en la primera línea de el archivo.tex de su respectivo libro
%%	%%	febrero 2018
%%	Autor: Imer A. Robles Rodriguez - imer.robles@correo.uis.edu.co
%%
%%	Archivo que contiene la mayoria de la configuracion complementaria
%%	que no pudo ser agregada en la clase 'icontecUIS.cls'
%%	incluirlo en la primera línea de el archivo.tex de su respectivo libro
%%	%%	febrero 2018
%%	Autor: Imer A. Robles Rodriguez - imer.robles@correo.uis.edu.co
%%
%%	Archivo que contiene la mayoria de la configuracion complementaria
%%	que no pudo ser agregada en la clase 'icontecUIS.cls'
%%	incluirlo en la primera línea de el archivo.tex de su respectivo libro
%%	%%	febrero 2018
%%	Autor: Imer A. Robles Rodriguez - imer.robles@correo.uis.edu.co
%%
%%	Archivo que contiene la mayoria de la configuracion complementaria
%%	que no pudo ser agregada en la clase 'icontecUIS.cls'
%%	incluirlo en la primera línea de el archivo.tex de su respectivo libro
%%	\input{path/to/structure}
%%
%%	Nota 1. en bibstyle=paht/to/icontec-imer, corregir el paht/to/icontec-imer 
%%
%%	Nota 2. para usar la cita icontec se usa el comando \footcite{}, en lugar del típico \cite{}

\documentclass[12pt,oneside,onecolumn,final,openright]{icontecUIS}
\usepackage[utf8]{inputenc}
\usepackage{lmodern}

\usepackage{csquotes}
\usepackage[spanish,es-tabla]{babel}
\usepackage[bibstyle=bib/icontec-imer,citestyle=verbose-trad2,sortcites=true,maxcitenames=3,maxbibnames=9,backend=biber]{biblatex} %


%--------Codigos para la caligrafia, tipos de letras%---------------
\usepackage{textcomp} %Paquete para algunos caracteres especiales
\usepackage[T1]{fontenc}
\usepackage[scaled]{uarial}
%\renewcommand{\familydefault}{\sfdefault} % sans serfi default
\renewcommand{\familydefault}{\rmdefault} % roman default
%\renewcommand{\baselinestretch}{1.5}
\usepackage{setspace}
\usepackage{amsmath,amsfonts}
\usepackage{graphicx}
\usepackage{subfig}
\usepackage{float}
\usepackage{booktabs}

\usepackage{datetime}


\usepackage[pdftex,%unicode=true,%pdftex,
pdfauthor={Imer A. Robles Rodríguez},
pdftitle={Tesis Pregrado},
%pdfsubject={The Subject},
% pdfkeywords={},
pdfproducer={Latex with hyperref, or other system},
pdfcreator={pdflatex, or other tool},
hidelinks,%=true,
bookmarks=true,
linktoc=all]{hyperref}


\usepackage{bookmark}

\usepackage{geometry}
\geometry{
	papersize = {216mm, 279mm}, % tamaño de papel carta colombia
	top = 4cm,
	left = 4cm,
	bottom = 3cm,
	right = 2cm,
	%% configurando notas la margen, encabezado y pie
%	headsep = 0.5em,
%	head = 0em,
	foot = 1cm,
	nomarginpar, nohead,
}


\hyphenation{op-tical net-works semi-conduc-tor}

\usepackage{titlesec}
\titleformat{\section}%[runin]
{\raggedright \normalfont\bfseries\uppercase} %formato
{\thesection.} %label
{0.5em}%separacion horizontal
{} %antes code []% despues code
\titlespacing{\section}{0pt}{*4}{*1}
\titleformat{\subsection}[runin]
{\raggedright \normalfont \bfseries \uppercase} %formato \fontsize{12pt}{12.2pt}
{\thesubsection.}{0.2em}{} %antes code 
[{\newline}]% despues code
\titlespacing{\subsection}{0pt}{*4}{*1}
\titleformat{\subsubsection}[runin]
{\raggedright \large \bfseries} %formato
{\thesubsubsection.} %label
{0.5em}%separacion horizontal
{} %antes code 
%[\vspace{-24pt}]% despues code

%%	comando definido para hacer fácil la inclusión de figuras
%%	\figura{ubicacion-de-figura/imagen}{nombre/caption}{ancho a usar en imagen/tamaño entre 0-1}{fuente/origen de la figura, si es propia dejar vacío}
\newcommand{\figura}[4]{
	\begin{figure}[H]
		\begin{minipage}{\textwidth}
			\caption[#2]{\raggedright #2}
			\label{fig:#2}
			\begin{center}
				\includegraphics[width=#3\textwidth]{#1}\\
			\end{center}
			#4
		\end{minipage}
	\end{figure}
}

\setlength{\parindent}{0em} % sangria (identacion) en 0
\addbibresource{Anexos/LATEX/source/bib/referencias.bib}

\usepackage{lipsum}
\usepackage{float}
\usepackage[utf8]{inputenc}
\usepackage[spanish]{babel}
\usepackage{newunicodechar}
\newunicodechar{fi}{fi}
\newunicodechar{ff}{ff}
%%
%%	Nota 1. en bibstyle=paht/to/icontec-imer, corregir el paht/to/icontec-imer 
%%
%%	Nota 2. para usar la cita icontec se usa el comando \footcite{}, en lugar del típico \cite{}

\documentclass[12pt,oneside,onecolumn,final,openright]{icontecUIS}
\usepackage[utf8]{inputenc}
\usepackage{lmodern}

\usepackage{csquotes}
\usepackage[spanish,es-tabla]{babel}
\usepackage[bibstyle=bib/icontec-imer,citestyle=verbose-trad2,sortcites=true,maxcitenames=3,maxbibnames=9,backend=biber]{biblatex} %


%--------Codigos para la caligrafia, tipos de letras%---------------
\usepackage{textcomp} %Paquete para algunos caracteres especiales
\usepackage[T1]{fontenc}
\usepackage[scaled]{uarial}
%\renewcommand{\familydefault}{\sfdefault} % sans serfi default
\renewcommand{\familydefault}{\rmdefault} % roman default
%\renewcommand{\baselinestretch}{1.5}
\usepackage{setspace}
\usepackage{amsmath,amsfonts}
\usepackage{graphicx}
\usepackage{subfig}
\usepackage{float}
\usepackage{booktabs}

\usepackage{datetime}


\usepackage[pdftex,%unicode=true,%pdftex,
pdfauthor={Imer A. Robles Rodríguez},
pdftitle={Tesis Pregrado},
%pdfsubject={The Subject},
% pdfkeywords={},
pdfproducer={Latex with hyperref, or other system},
pdfcreator={pdflatex, or other tool},
hidelinks,%=true,
bookmarks=true,
linktoc=all]{hyperref}


\usepackage{bookmark}

\usepackage{geometry}
\geometry{
	papersize = {216mm, 279mm}, % tamaño de papel carta colombia
	top = 4cm,
	left = 4cm,
	bottom = 3cm,
	right = 2cm,
	%% configurando notas la margen, encabezado y pie
%	headsep = 0.5em,
%	head = 0em,
	foot = 1cm,
	nomarginpar, nohead,
}


\hyphenation{op-tical net-works semi-conduc-tor}

\usepackage{titlesec}
\titleformat{\section}%[runin]
{\raggedright \normalfont\bfseries\uppercase} %formato
{\thesection.} %label
{0.5em}%separacion horizontal
{} %antes code []% despues code
\titlespacing{\section}{0pt}{*4}{*1}
\titleformat{\subsection}[runin]
{\raggedright \normalfont \bfseries \uppercase} %formato \fontsize{12pt}{12.2pt}
{\thesubsection.}{0.2em}{} %antes code 
[{\newline}]% despues code
\titlespacing{\subsection}{0pt}{*4}{*1}
\titleformat{\subsubsection}[runin]
{\raggedright \large \bfseries} %formato
{\thesubsubsection.} %label
{0.5em}%separacion horizontal
{} %antes code 
%[\vspace{-24pt}]% despues code

%%	comando definido para hacer fácil la inclusión de figuras
%%	\figura{ubicacion-de-figura/imagen}{nombre/caption}{ancho a usar en imagen/tamaño entre 0-1}{fuente/origen de la figura, si es propia dejar vacío}
\newcommand{\figura}[4]{
	\begin{figure}[H]
		\begin{minipage}{\textwidth}
			\caption[#2]{\raggedright #2}
			\label{fig:#2}
			\begin{center}
				\includegraphics[width=#3\textwidth]{#1}\\
			\end{center}
			#4
		\end{minipage}
	\end{figure}
}

\setlength{\parindent}{0em} % sangria (identacion) en 0
\addbibresource{Anexos/LATEX/source/bib/referencias.bib}

\usepackage{lipsum}
\usepackage{float}
\usepackage[utf8]{inputenc}
\usepackage[spanish]{babel}
\usepackage{newunicodechar}
\newunicodechar{fi}{fi}
\newunicodechar{ff}{ff}
%%
%%	Nota 1. en bibstyle=paht/to/icontec-imer, corregir el paht/to/icontec-imer 
%%
%%	Nota 2. para usar la cita icontec se usa el comando \footcite{}, en lugar del típico \cite{}

\documentclass[12pt,oneside,onecolumn,final,openright]{icontecUIS}
\usepackage[utf8]{inputenc}
\usepackage{lmodern}

\usepackage{csquotes}
\usepackage[spanish,es-tabla]{babel}
\usepackage[bibstyle=bib/icontec-imer,citestyle=verbose-trad2,sortcites=true,maxcitenames=3,maxbibnames=9,backend=biber]{biblatex} %


%--------Codigos para la caligrafia, tipos de letras%---------------
\usepackage{textcomp} %Paquete para algunos caracteres especiales
\usepackage[T1]{fontenc}
\usepackage[scaled]{uarial}
%\renewcommand{\familydefault}{\sfdefault} % sans serfi default
\renewcommand{\familydefault}{\rmdefault} % roman default
%\renewcommand{\baselinestretch}{1.5}
\usepackage{setspace}
\usepackage{amsmath,amsfonts}
\usepackage{graphicx}
\usepackage{subfig}
\usepackage{float}
\usepackage{booktabs}

\usepackage{datetime}


\usepackage[pdftex,%unicode=true,%pdftex,
pdfauthor={Imer A. Robles Rodríguez},
pdftitle={Tesis Pregrado},
%pdfsubject={The Subject},
% pdfkeywords={},
pdfproducer={Latex with hyperref, or other system},
pdfcreator={pdflatex, or other tool},
hidelinks,%=true,
bookmarks=true,
linktoc=all]{hyperref}


\usepackage{bookmark}

\usepackage{geometry}
\geometry{
	papersize = {216mm, 279mm}, % tamaño de papel carta colombia
	top = 4cm,
	left = 4cm,
	bottom = 3cm,
	right = 2cm,
	%% configurando notas la margen, encabezado y pie
%	headsep = 0.5em,
%	head = 0em,
	foot = 1cm,
	nomarginpar, nohead,
}


\hyphenation{op-tical net-works semi-conduc-tor}

\usepackage{titlesec}
\titleformat{\section}%[runin]
{\raggedright \normalfont\bfseries\uppercase} %formato
{\thesection.} %label
{0.5em}%separacion horizontal
{} %antes code []% despues code
\titlespacing{\section}{0pt}{*4}{*1}
\titleformat{\subsection}[runin]
{\raggedright \normalfont \bfseries \uppercase} %formato \fontsize{12pt}{12.2pt}
{\thesubsection.}{0.2em}{} %antes code 
[{\newline}]% despues code
\titlespacing{\subsection}{0pt}{*4}{*1}
\titleformat{\subsubsection}[runin]
{\raggedright \large \bfseries} %formato
{\thesubsubsection.} %label
{0.5em}%separacion horizontal
{} %antes code 
%[\vspace{-24pt}]% despues code

%%	comando definido para hacer fácil la inclusión de figuras
%%	\figura{ubicacion-de-figura/imagen}{nombre/caption}{ancho a usar en imagen/tamaño entre 0-1}{fuente/origen de la figura, si es propia dejar vacío}
\newcommand{\figura}[4]{
	\begin{figure}[H]
		\begin{minipage}{\textwidth}
			\caption[#2]{\raggedright #2}
			\label{fig:#2}
			\begin{center}
				\includegraphics[width=#3\textwidth]{#1}\\
			\end{center}
			#4
		\end{minipage}
	\end{figure}
}

\setlength{\parindent}{0em} % sangria (identacion) en 0
\addbibresource{Anexos/LATEX/source/bib/referencias.bib}

\usepackage{lipsum}
\usepackage{float}
\usepackage[utf8]{inputenc}
\usepackage[spanish]{babel}
\usepackage{newunicodechar}
\newunicodechar{fi}{fi}
\newunicodechar{ff}{ff}
%%
%%	Nota 1. en bibstyle=paht/to/icontec-imer, corregir el paht/to/icontec-imer 
%%
%%	Nota 2. para usar la cita icontec se usa el comando \footcite{}, en lugar del típico \cite{}

\documentclass[12pt,oneside,onecolumn,final,openright]{icontecUIS}
\usepackage[utf8]{inputenc}
\usepackage{lmodern}

\usepackage{csquotes}
\usepackage[spanish,es-tabla]{babel}
\usepackage[bibstyle=bib/icontec-imer,citestyle=verbose-trad2,sortcites=true,maxcitenames=3,maxbibnames=9,backend=biber]{biblatex} %


%--------Codigos para la caligrafia, tipos de letras%---------------
\usepackage{textcomp} %Paquete para algunos caracteres especiales
\usepackage[T1]{fontenc}
\usepackage[scaled]{uarial}
%\renewcommand{\familydefault}{\sfdefault} % sans serfi default
\renewcommand{\familydefault}{\rmdefault} % roman default
%\renewcommand{\baselinestretch}{1.5}
\usepackage{setspace}
\usepackage{amsmath,amsfonts}
\usepackage{graphicx}
\usepackage{subfig}
\usepackage{float}
\usepackage{booktabs}

\usepackage{datetime}


\usepackage[pdftex,%unicode=true,%pdftex,
pdfauthor={Imer A. Robles Rodríguez},
pdftitle={Tesis Pregrado},
%pdfsubject={The Subject},
% pdfkeywords={},
pdfproducer={Latex with hyperref, or other system},
pdfcreator={pdflatex, or other tool},
hidelinks,%=true,
bookmarks=true,
linktoc=all]{hyperref}


\usepackage{bookmark}

\usepackage{geometry}
\geometry{
	papersize = {216mm, 279mm}, % tamaño de papel carta colombia
	top = 4cm,
	left = 4cm,
	bottom = 3cm,
	right = 2cm,
	%% configurando notas la margen, encabezado y pie
%	headsep = 0.5em,
%	head = 0em,
	foot = 1cm,
	nomarginpar, nohead,
}


\hyphenation{op-tical net-works semi-conduc-tor}

\usepackage{titlesec}
\titleformat{\section}%[runin]
{\raggedright \normalfont\bfseries\uppercase} %formato
{\thesection.} %label
{0.5em}%separacion horizontal
{} %antes code []% despues code
\titlespacing{\section}{0pt}{*4}{*1}
\titleformat{\subsection}[runin]
{\raggedright \normalfont \bfseries \uppercase} %formato \fontsize{12pt}{12.2pt}
{\thesubsection.}{0.2em}{} %antes code 
[{\newline}]% despues code
\titlespacing{\subsection}{0pt}{*4}{*1}
\titleformat{\subsubsection}[runin]
{\raggedright \large \bfseries} %formato
{\thesubsubsection.} %label
{0.5em}%separacion horizontal
{} %antes code 
%[\vspace{-24pt}]% despues code

%%	comando definido para hacer fácil la inclusión de figuras
%%	\figura{ubicacion-de-figura/imagen}{nombre/caption}{ancho a usar en imagen/tamaño entre 0-1}{fuente/origen de la figura, si es propia dejar vacío}
\newcommand{\figura}[4]{
	\begin{figure}[H]
		\begin{minipage}{\textwidth}
			\caption[#2]{\raggedright #2}
			\label{fig:#2}
			\begin{center}
				\includegraphics[width=#3\textwidth]{#1}\\
			\end{center}
			#4
		\end{minipage}
	\end{figure}
}

\setlength{\parindent}{0em} % sangria (identacion) en 0
\addbibresource{Anexos/LATEX/source/bib/referencias.bib}

\usepackage{lipsum}
\usepackage[utf8]{inputenc}
\usepackage[spanish]{babel}
\usepackage{newunicodechar}
\newunicodechar{fi}{fi}
\newunicodechar{ff}{ff}