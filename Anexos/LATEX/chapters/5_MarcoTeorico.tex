\chapter{MARCO TEÓRICO}
\section{MARCO CONCEPTUAL}
\subsection{LeapMotion}
.
\subsection{Goniometria}
.
\subsection{Análisis de puesto de trabajo}
.
\subsection{Enfermedades y lesiones}
 Un Desorden Músculo Esquelético (DME) es una lesión física originada por trauma acumulado que se desarrolla gradualmente sobre un período de tiempo; como resultado de repetidos esfuerzos sobre una parte específica del sistema músculo esquelético. \footcite[]{MinisterioCONTINUA}. A continuación se presentan las enfermedades prevalentes en los puestos de trabajo para manos, muñeca y antebrazo

\subsubsection{Tenosinovitis De Quervain}
\paragraph{¿Qué es?}
La Tenosinovitis De Quervainn (TDQ) también conocida como tenosinovitis estiloides radiales, consiste en la inflamación de los tendones del pulgar a causa de movimientos repetitivos.

\begin{figure}[H]
    \centering
    \includegraphics[width=0.7\textwidth]{Anexos/LATEX/chapters/images/TDQ.jpg}
    \caption{Anatomía de la mano con TDQ \\\textbf{Fuente:} https://g.co/kgs/BpkB4A}
    \label{TDQ}
\end{figure}

Esta enfermedad se manifiesta como una inflamación que produce una estenosis del canal osteofibrososinovial situado en la estiloides radial por el que discurren los tendones del abductor largo y extensor corto del pulgar. Se produce al combinar agarres con giros o desviaciones cubitales y radiales repetidas o forzadas de la mano.\footcite[2]{TenosinovitisDelPulgarDDCENFERMEDADESTME}

De acuerdo al Instituto Nacional de Seguridad e Higiene en el Trabajo de España, la Enfermedad de Quervain (CIE-9 MC 727.04) posee un tiempo estándar de incapacidad temporal de 20 días.\footcite[6]{TenosinovitisDelPulgarDDCENFERMEDADESTME}
\paragraph{Prevención}
Se aconseja no combinar agarres con giros o desviaciones cubitales y radiales repetidas. Para las situaciones de oficina que involucran un mouse, es necesario evitar realizar desplazamientos girando la muñeca, el movimiento adecuado debe ser desplazar el brazo en su totalidad desde el hombro
\subsubsection{Sindrome del Túnel del Carpo}
\paragraph{¿Qué es?}
Existe un espacio en la muñeca llamado túnel del carpo, a través del cual pasan el nervio mediano y nueve tendones flexores que van desde el antebrazo hacia la mano. El Síndrome del Túnel del Carpo (STC) es una condición producida por la compresión del Nervio Mediano, a nivel de la muñeca. Esta compresión produce entumecimiento, hormigueo y dolor en la mano, dedos y ocasionalmente en el brazo. \footcite{SindromeCarpiano}

\begin{figure}[H]
    \centering
    \includegraphics[width=0.7\textwidth]{Anexos/LATEX/chapters/images/STC.jpg}
    \caption{Anatomía de la mano con STC \\\textbf{Fuente:} https://g.co/kgs/MdPBTe}
    \label{STC}
\end{figure}

El STC se presenta cuando se aumenta la presión dentro del túnel por cualquier proceso inflamatorio, comprimiendo el nervio, el cual es una estructura muy sensible a los aumentos de presión. Cuando la presión dentro del túnel es muy alta y altera la función normal del nervio, aparecen rigidez, hormigueo y dolor en la mano y los dedos.\footcite{SindromeCarpiano}
\paragraph{Prevención}
Es recomendable informar al trabajador, entrenándolo para que aquellas posturas o movimientos peligrosos sean evitados durante el desarrollo de su labor, además, el buen diseño de las herramientas, utensilios y del puesto de trabajo ayudan a conseguir la relajación de la mano y de la muñeca.\footcite{SindromeTratarlo}
\subsection{Legislación Vigente}
\section{MARCO REFERENCIAL}
\subsection{Modelos de valoración del riesgo  contexto nacional}
\subsubsection{Normas GATI-SO}
Las Guías de Atención Integral de Salud Ocupacional Basadas en la Evidencia (GATI-SO) nacen a partir de un plan de trabajo propuesto por la dirección general de riesgos profesionales del ministerio de la protección social con el objetivo de incrementar el diagnóstico y prevenir las enfermedades profesionales de
mayor prevalencia en Colombia.\footcite[6]{MinisterioCONTINUA}

Las características de los factores de riesgo ocupacional que han demostrado estar asociados con la aparición del \textbf{STC} son las siguientes:\footcite[45]{MinisterioCONTINUA}
\begin{itemize}
\item Posturas en flexión y extensión de dedos, mano y muñeca, así como, la desviación ulnar o radial que implique agarre, pronación y supinación combinada con el movimiento repetitivo en ciclos de trabajo
\item Fuerza ejercida en trabajo dinámico por manipulación de pesos en extensión y flexión de los dedos y la mano
\item Vibración segmentaría derivada del uso de herramientas vibratorias
\end{itemize}
Las características de los factores de riesgo ocupacional que han demostrado estar asociados con la aparición del \textbf{TDQ} son las siguientes:\footcite[45]{MinisterioCONTINUA}
\begin{itemize}
\item Postura forzada de muñeca asociada a movimiento de alta repetición (ciclos de tiempo menores a 30 segundos o 50 \% del ciclo gastado.
\end{itemize}
Adicionalmente se mencionan las siguientes conclusiones:\footcite[46]{MinisterioCONTINUA}
\begin{itemize}
\item Existe evidencia de que las posturas asumidas de codo, antebrazo y mano se asocian con mayor frecuencia a los desórdenes de trauma acumulativo en población trabajadora.
\item Existe evidencia de que el movimiento repetitivo se asocia con mayor frecuencia a los desórdenes de trauma acumulativo en población trabajadora.
\item Existe evidencia de que la fuerza se asocia con mayor frecuencia a los desórdenes de trauma acumulativo en población trabajadora.
\end{itemize}



\section{Modelos de ergonomía para brazos y muñecas en el contexto internacional}
\subsubsection{}
\section{MARCO TECNOLÓGICO}