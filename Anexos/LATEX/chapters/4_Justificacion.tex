\chapter{JUSTIFICACIÓN}
% Síntomas del problema (Variables dependientes)
% 1. Los empleados que se enferman por DME está en aumento
En los últimos 10 años se ha evidenciado un importante crecimiento en el uso de computadores como elemento de trabajo principal en las oficinas, asimismo, el DANE (Departamento Administrativo Nacional de Estadística)  muestra que 99.4\% de las empresas en los sectores de industria manufacturera, comercio y servicios poseen al menos un computador y de este total general el 75.83 \% usan el computador de forma diaria durante aproximadamente 6 horas de las 8 horas laborales establecidas.\footcite[]{Dane2013IndicadoresEmpresas}
% 2. Los métodos actuales de diagnóstico no son efectivos
A pesar de que las empresas Administradoras de Riesgos Laborales (ARL) realizan revisiones de puestos de trabajo, el Informe Colombiano de Enfermedades Profesionales para el 2017 indica que la tasa de enfermedades laborales para el país fue de 94,7 por cada 100.000 trabajadores, (FASECOLDA). Teniéndose de esta manera, uno de los aspectos más importantes que justifican el proyecto 'NOMBREDELPROYECTO', el cual gira entorno a la reducción de Desórdenes Músculo Esqueléticos (DME) a causa de exigencias biomecánicas no supervisadas en los puestos de trabajo.
% 3. Las incapacidades a causa de DME generan perdidas monetarias
En las empresas colombianas actualmente se invierten casi \$ 2 billones en sistemas de riesgos laborales, un porcentaje de ese total está destinado a las prestaciones asistenciales y el otro a la promoción y prevención de los accidentes laborales, siendo mas alto el porcentaje a la mitigación de estos  (REVISTA DINERO), y a pesar que en Colombia ha habido una reducción de 7\% en los accidentes laborales en 2017 frente a 2016, según el análisis de la Federación de Aseguradores Colombianos (Fasecolda)


%4. El nivel de calidad de vida de los empleados con DME disminuye ya que son lesiones permanentes

% Causas del problema (Variables independientes)
% 1. Desconocimiento por parte de los usuarios de equipos de oficina de las buenas prácticas ergonómicas en el trabajo

% 2. Las revisiones de puesto de trabajo llevadas a cabo por las ARL son insuficientes o inexistentes, además solo se llevan a cabo durante un periodo de tiempo determinado en vez de un monitoreo constante

% Diagnóstico (Relato de la situación actual)
% Los usuarios de equipo de oficina en Colombia se encuentran expuestos a lesiones físicas originadas por traumas acumulados que se desarrollan gradualmente sobre un período de tiempo a causa de movimientos repetitivos y malas posturas

% Pronóstico (Qué pasa si el diagnóstico continúa)
% Considerando que la cantidad de trabajadores que utilizan equipos de oficina se encuentra en aumento, cada vez habrán más trabajadores lesionados y comenzará a ser una porcentaje importante de la población Colombiana, lo anterior conduce a una saturación en los centros hospitalarios por tratamientos de desordenes musculo esqueléticos e inversiones millonarias en sistemas de riesgos laborales para suplir esta necesidad.

% Control al pronóstico (Qué hacer para que el pronostico no se cumpla)
% Por esta razón, se desea contar con una herramienta que identifique a tiempo los factores de riesgo y suministre un monitoreo constante en el puesto de trabajo para evitar lesiones que puedan desencadenar este tipo de traumatismos.

% Partiendo de dicha necesidad, se decide poner en marcha el proyecto ‘NOMBRE DEL PROYECTO’ en su primera versión, en donde se propone diseñar, desarrollar y probar un prototipo de software como soporte a la gestión de la revisión de puestos de trabajo para los miembros superiores al utilizar un mouse y teclado.



En la siguiente tabla se puede 

\begin{table}[H]
\begin{tabular}{lcc}
\textbf{Departamento} & \multicolumn{1}{l}{\textbf{Total de trabajadores}} & \multicolumn{1}{l}{\textbf{Total de enfermedades}} \\
Atlantico             & -18\%                                              & 13\%                                               \\
Antioquia             & -19\%                                              & 9\%                                                \\
Bogotá                & -15\%                                              & -23\%                                              \\
Bolivar               & -14\%                                              & -75\%                                              \\
Casanare              & -24\%                                              & 33\%                                               \\
Meta                  & -16\%                                              & -22\%                                              \\
Norte de Santander    & -17\%                                              & 14\%                                               \\
Arauca                & -21                                                & 50\%                                               \\
Boyaca                & -20                                                & -29\%                                             
\end{tabular}
\end{table}

Esto ha llevado a las empresas en el país a que, entre 2016 y 2017, hayan invertido en el sistema de riesgos laborales recursos que llegan casi a los \$2 billones anuales 


En las normas GATISO se recomienda que para la identificación de factores de riesgo ocupacional asociados con los DME, se utilicen estrategias como:
\begin{itemize}
    \item Auto reportes, inspecciones estructuradas que sirvan como diagnóstico precoz de las condiciones de riesgo, posteriormente se debe utilizar listas de chequeo orientadas al reconocimiento de peligros como posturas, fuerzas, repetición, vibración y bajas temperatura
    \item Encuestas de morbilidad sentida de los trabajadores expuestos
    \item Estudio de casos previos reportados en la empresa
\end{itemize}

% Añadir justificación desde:
% 1. Perspectiva Monetaria: Las empresas pierden plata
% 2. Enfoque humano: Las personas quedan lesionadas PERMANENTEMENTE teniendo menor calidad de vida
% 3. Gastos en Salud: El estado gasta X plata cubriendo DME
% 4. Los DME son generados principalmente a partir de movimientos repetitivos VER https://www.epssura.com/guias/guias_mmss.pdf || Página 19 - Tabla 1

por esa razón se pretende estudiar los factores de riesgo laboral que influyen en el desarrollo de este tipo de lesiones del tren superior y extremidades, y la influencia de los factores del trabajo como determinados movimientos repetitivos y posturas
