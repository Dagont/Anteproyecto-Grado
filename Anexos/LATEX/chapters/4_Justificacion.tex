\chapter{JUSTIFICACIÓN}
En los últimos 10 años se ha evidenciado un importante crecimiento en el uso de tecnología (Computadores, tabletas, celulares etc.) como elemento principal de trabajo, así mismo, el DANE (Departamento Administrativo Nacional de Estadística)  muestra que 99.4\% de las empresas en los sectores de industria manufacturera, comercio y servicios poseen al menos un computador y de este total general el 75.83 \% usan el computador de forma diaria durante aproximadamente 6 horas de las 8 horas laborales establecidas.\parencite[]{Dane2013IndicadoresEmpresas} a consecuencia de lo anterior, los trabajadores realizan tareas repetitivas durante su jornada laboral, lo que los expone a lesiones físicas originadas por traumas acumulados que se desarrollan gradualmente sobre un período de tiempo a causa de movimientos repetitivos y malas posturas, por esta razón, el Ministerio del trabajo se vió incentivado a implementar el Decreto 1072 de 2015 en Seguridad y salud en el trabajo (SG-SST) exponiendo en su Artículo 2.2.4.1.3, que: \textit{Por ley las empresas deben adoptar un sistema de gestión de seguridad y salud en el trabajo sustentado bajo el Artículo 2.2.4.1.6}, donde este último trata de accidentes y enfermedades laborales, todo esto con la intención de fortalecer las empresas en temas de seguridad y salud en el trabajo.  


A partir de lo anterior, las empresas colombianas invierten casi \$ 2 billones de pesos en sistemas de riesgos laborales, en donde para el 2017, el porcentaje más alto (36\%) se dirige a la promoción y prevención de los accidentes laborales. 
\begin{table}[H]
\resizebox{1\textwidth}{!}{%
\begin{tabular}{r|r|r|r|r|r}
\cline{2-5}
\multicolumn{1}{l|}{}                            & \multicolumn{2}{c|}{2016}                                          & \multicolumn{2}{c|}{2017}                                          & \multicolumn{1}{c}{}                 \\ \cline{2-6} 
\multicolumn{1}{l|}{}                            & \multicolumn{1}{c|}{Inversión} & \multicolumn{1}{c|}{\% del Total} & \multicolumn{1}{c|}{Inversión} & \multicolumn{1}{c|}{\% del Total} & \multicolumn{1}{c|}{Variación \%}    \\ \hline
\multicolumn{1}{|r|}{Prestaciones Asistenciales} & \$ 679.454                     & 36\%                              & \$ 655.811                     & 34\%                              & \multicolumn{1}{r|}{-3,48\%}         \\ \hline
\multicolumn{1}{|r|}{Promoción y Prevención}     & \$ 665.948                     & 35\%                              & \$ 699.788                     & 36\%                              & \multicolumn{1}{r|}{5,08\%}          \\ \hline
\multicolumn{1}{|r|}{Prestaciones económicas}    & \$ 559.419                     & 29\%                              & \$ 591.743                     & 30\%                              & \multicolumn{1}{r|}{5,78\%}          \\ \hline
Total                                            & \textbf{\$ 1.904.821}          & \textbf{100\%}                    & \textbf{\$ 1.947.342}          & \textbf{100\%}                    & \multicolumn{1}{r|}{\textbf{2,23\%}} \\ \cline{2-6} 
\end{tabular}
}
\centering
\footnotesize{(Cifras en Miles de Millones), Fuente: \cite{RevistaDinero2018ReduciendoColombianas}}
\caption{Inversión en Sistemas de Riesgos Laborales (2016 vs 2017)}
\label{InversionSRL}
\end{table}

Como se observa en la Tabla \ref{InversionSRL}, en el 2017 el porcentaje asignado para la mitigación de enfermedades laborales crece en un 5\% con respecto al año anterior, sobrepasando el enfoque que se tenía a Prestaciones Asistenciales, según el análisis de la Federación de Aseguradores Colombianos (Fasecolda) a causa de esta redirección del presupuesto se impulsó una reducción de 7\% en los accidentes laborales en 2017 frente a 2016.\parencite{RevistaDinero2018ReduciendoColombianas} 

Sin embargo, a pesar de que el presupuesto asignado para las enfermedades laborales fue incrementado y que las empresas Administradoras de Riesgos Laborales (ARL) están obligadas a realizar revisiones de puestos de trabajo, el Informe Colombiano de Enfermedades Profesionales para el 2017 indica que la tasa de enfermedades laborales para el país fue de 94,7 por cada 100.000 trabajadores, (FASECOLDA). 

Desde la perspectiva del trabajador, una persona que llega sufrir de Desórdenes Músculo Esqueléticos (DME), no solo queda incapacitada para laborar en el mismo trabajo si no en cualquier otro que necesite el uso de miembros distales superiores afectados, de la misma manera, se dificultan las tareas diarias del lesionado, perdiendo calidad de vida, forzando la inclusión en su rutina diaria elementos ortopédicos, terapia física, medicamentos y posibles cirugías para la recuperación, en este punto la compensación laboral económica no suple todo lo que se perdió. 

Según datos de la Organización Internacional del
Trabajo (OIT), las enfermedades laborales y los accidentes
relacionados con el trabajo ocasionan dos millones de
muertes en el mundo, cuyo costo para la economía global asciende
a 1,4\% del Producto Interno Bruto Global. Adicional al
pago de indemnizaciones, asumidos principalmente por el
sistema sanitario, la sociedad en su conjunto debe afrontar
otros gastos como consecuencia de estos eventos, entre
los cuales se cuentan: disminución de la competitividad, la
jubilación anticipada, el ausentismo laboral, el desempleo
y la disminución de los ingresos del hogar \parencite{OrganizacionInternacionaldelTrabajo2003OIT:Billones}


Por esa razón se pretende estudiar los factores de riesgo laboral que influyen y aceleran el desarrollo de lesiones de los miembros distales superiores, siendo de ayuda al crecimiento tecnológico sin afectar la esencia ergonómica del trabajador.


% Síntomas del problema (Variables dependientes)
% 1. Los empleados que se enferman por DME está en aumento
% 1. Perspectiva Monetaria: Las empresas pierden plata
%Al momento de ocurrir un siniestro laboral ya sea por accidente o enfermedad, las empresas para solventarlo junto a la ARL tiene que hacer un gasto en tiempo en documentación e investigación del siniestro, para que la ARL pueda considerar costos médicos directos e indirectos relacionados con el accidente o la enfermedad laboral, en caso que la empresa no cumpla las legislación vigente en la prevención y mitigación esta tendrá que incurrir en los gastos del empleado. 

% 2. Los métodos actuales de diagnóstico no son efectivos

%Teniéndose de esta manera, uno de los aspectos más importantes que justifican el proyecto 'NOMBREDELPROYECTO', el cual gira entorno a la reducción de Desórdenes Músculo Esqueléticos (DME) a causa de exigencias biomecánicas no supervisadas en los puestos de trabajo.
% 3. Las incapacidades a causa de DME generan perdidas monetarias

% 2. Enfoque humano: Las personas quedan lesionadas PERMANENTEMENTE teniendo menor calidad de vida

%4. El nivel de calidad de vida de los empleados con DME disminuye ya que son lesiones permanentes

% Causas del problema (Variables independientes)
% 1. Desconocimiento por parte de los usuarios de equipos de oficina de las buenas prácticas ergonómicas en el trabajo

% 2. Las revisiones de puesto de trabajo llevadas a cabo por las ARL son insuficientes o inexistentes, además solo se llevan a cabo durante un periodo de tiempo determinado en vez de un monitoreo constante

% Diagnóstico (Relato de la situación actual)
% Los usuarios de equipo de oficina en Colombia se encuentran expuestos a lesiones físicas originadas por traumas acumulados que se desarrollan gradualmente sobre un período de tiempo a causa de movimientos repetitivos y malas posturas

% Pronóstico (Qué pasa si el diagnóstico continúa)
% Considerando que la cantidad de trabajadores que utilizan equipos de oficina se encuentra en aumento, cada vez habrán más trabajadores lesionados y comenzará a ser una porcentaje importante de la población Colombiana, lo anterior conduce a una saturación en los centros hospitalarios por tratamientos de desordenes musculo esqueléticos e inversiones millonarias en sistemas de riesgos laborales para suplir esta necesidad.

% Control al pronóstico (Qué hacer para que el pronostico no se cumpla)
% Por esta razón, se desea contar con una herramienta que identifique a tiempo los factores de riesgo y suministre un monitoreo constante en el puesto de trabajo para evitar lesiones que puedan desencadenar este tipo de traumatismos.

% Partiendo de dicha necesidad, se decide poner en marcha el proyecto ‘NOMBRE DEL PROYECTO’ en su primera versión, en donde se propone diseñar, desarrollar y probar un prototipo de software como soporte a la gestión de la revisión de puestos de trabajo para los miembros superiores al utilizar un mouse y teclado.



%En las normas GATISO se recomienda que para la identificación de factores de riesgo ocupacional asociados con los DME, se utilicen estrategias como:
%\begin{itemize}
%    \item Auto reportes, inspecciones estructuradas que sirvan como diagnóstico precoz de las condiciones de riesgo, posteriormente se debe utilizar listas de chequeo orientadas al reconocimiento de peligros como posturas, fuerzas, repetición, vibración y bajas temperatura
    %\item Encuestas de morbilidad sentida de los trabajadores expuestos
    %\item Estudio de casos previos reportados en la empresa
%\end{itemize}

% Añadir justificación desde:
% 1. Perspectiva Monetaria: Las empresas pierden plata
% 2. Enfoque humano: Las personas quedan lesionadas PERMANENTEMENTE teniendo menor calidad de vida
% 3. Gastos en Salud: El estado gasta X plata cubriendo DME
% 4. Los DME son generados principalmente a partir de movimientos repetitivos VER https://www.epssura.com/guias/guias_mmss.pdf || Página 19 - Tabla 1