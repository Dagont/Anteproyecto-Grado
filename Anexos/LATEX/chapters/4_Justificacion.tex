\chapter{JUSTIFICACIÓN}
En los últimos 10 años se ha evidenciado un importante crecimiento en el uso de computadores como elemento de trabajo principal en las oficinas, asimismo, el DANE (Departamento Administrativo Nacional de Estadística)  muestra que 99.4\% de las empresas en los sectores de industria manufacturera, comercio y servicios poseen al menos un computador y de este total general el 75.83 \% usan el computador de forma diaria durante aproximadamente 6 horas de las 8 horas laborales establecidas.\footcite[]{Dane2013IndicadoresEmpresas}

A pesar de que las empresas Administradoras de Riesgos Laborales (ARL) realizan revisiones de puestos de trabajo, el Informe Colombiano de Enfermedades Profesionales para el 2017 indica que la tasa de enfermedades laborales para el país fue de 94,7 por cada 100.000 trabajadores, (FASECOLDA). Teniéndose de esta manera, uno de los aspectos más importantes que justifican el proyecto 'NOMBREDELPROYECTO', el cual gira entorno a la reducción de Desórdenes Músculo Esqueléticos (DME) a causa de exigencias biomecánicas no supervisadas en los puestos de trabajo.

En las empresas colombianas actualmente se invierten casi \$ 2 billones en sistemas de riesgos laborales, un porcentaje de ese total está destinado a las prestaciones asistenciales y el otro a la promoción y prevención de los accidentes laborales, siendo mas alto el porcentaje a la mitigación de estos  (REVISTA DINERO), y a pesar que en Colombia ha habido una reducción de 7\% en los accidentes laborales en 2017 frente a 2016, según el análisis de la Federación de Aseguradores Colombianos (Fasecolda)

En la siguiente tabla se puede 


En las normas GATISO se recomienda que para la identificación de factores de riesgo ocupacional asociados con los DME, se utilicen estrategias como:
\begin{itemize}
    \item Auto reportes, inspecciones estructuradas que sirvan como diagnóstico precoz de las condiciones de riesgo, posteriormente se debe utilizar listas de chequeo orientadas al reconocimiento de peligros como posturas, fuerzas, repetición, vibración y bajas temperatura
    \item Encuestas de morbilidad sentida de los trabajadores expuestos
    \item Estudio de casos previos reportados en la empresa
\end{itemize}

% Añadir justificación desde:
% 1. Perspectiva Monetaria: Las empresas pierden plata
% 2. Enfoque humano: Las personas quedan lesionadas PERMANENTEMENTE teniendo menor calidad de vida
% 3. Gastos en Salud: El estado gasta X plata cubriendo DME
% 4. Los DME son generados principalmente a partir de movimientos repetitivos VER https://www.epssura.com/guias/guias_mmss.pdf || Página 19 - Tabla 1

por esa razón se pretende estudiar los factores de riesgo laboral que influyen en el desarrollo de este tipo de lesiones del tren superior y extremidades, y la influencia de los factores del trabajo como determinados movimientos repetitivos y posturas
