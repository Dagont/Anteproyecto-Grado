\chapter{OBJETIVOS}
\section{OBJETIVO GENERAL}

Desarrollar un prototipo de software que permita apoyar el proceso de revisión de puestos de trabajo a partir de una evaluación ergonómica de los miembros distales superiores utilizando un dispositivo de hardware que identifique y puntualice los vicios posturales durante las actividades laborales
%\paragraph{Propuesta 2}
%Crear a partir de una herramienta de hardware un prototipo de software, que permita a través de una interfaz de lenguaje natural, capturar los movimientos de las extremidades superiores en puestos de trabajo, que permita tener una evaluación dinámica y  constante de la pulcritud postural durante toda la ejecución de la actividad laboral. 
\section{OBJETIVOS ESPECÍFICOS}
\begin{itemize}
    \item Determinar las condiciones ergonómicas adecuadas para las muñecas y antebrazos a partir de una revisión sistemática de los estándares utilizados por especialistas de salud ocupacional en el contexto nacional e internacional para determinar vicios posturales, de esta manera precisando el modelo que más se adecúa al prototipo propuesto.
    \item Diseñar un modelo computacional para la interpretación de los datos generados por la herramienta de hardware LeapMotion.
    \item Adoptar una metodología que permita desarrollar de forma ágil un prototipo de software que evalúe la postura del antebrazo y mano. %en tiempo real. 
    \item Establecer un modelo de evaluación para los datos capturados por la herramienta que determine las posturas ergonómicamente incorrectas. 
\end{itemize}