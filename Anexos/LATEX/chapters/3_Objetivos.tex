\chapter{OBJETIVOS}
\section{OBJETIVO GENERAL}
Desarrollar un prototipo de software de escritorio que permita apoyar el proceso de revisión de puestos de trabajo a partir de una evaluación ergonómica de los miembros distales superiores utilizando un dispositivo de hardware que identifique y puntualice los vicios posturales durante las actividades laborales

Desarrollar un prototipo de software de escritorio que permita apoyar el proceso de revisión de puestos de trabajo a partir de una evaluación ergonómica de los miembros distales superiores, identificando y puntualizando los vicios posturales durante las actividades laborales con base en lo percibido por el dispositivo de hardware
 
\section{OBJETIVOS ESPECÍFICOS}
\begin{itemize}
    \item Determinar las condiciones ergonómicas adecuadas para las muñecas y antebrazos a partir de una revisión sistemática de los estándares utilizados por especialistas de salud ocupacional en el contexto nacional e internacional, de esta manera precisando el modelo que más se adecúa al prototipo propuesto.
    \item Diseñar un modelo computacional para la captura de los datos generados por la herramienta de hardware LeapMotion.
    %\item Diseñar un modelo vectorial que relacione los datos capturados y las posiciones ergonómicas adecuadas.
    \item Establecer un módulo de interpretación y evaluación para los datos capturados por la herramienta que determine las posturas ergonómicamente incorrectas.
    \item Adoptar una técnica de inteligencia artificial que permita reconocer vicios posturales.
    \item Establecer un modelo de verificación y evaluación para los resultados obtenidos.
\end{itemize}