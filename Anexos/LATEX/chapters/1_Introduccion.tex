\newpage
\chapter{INTRODUCCIÓN} 
La Dirección General de Riesgos Profesionales del Ministerio de la Protección Social publicó en el año 2004 el informe de enfermedades profesionales en Colombia 2001 – 2002, en el cual se define un plan de trabajo cuyo objetivo fundamental es incrementar el diagnóstico y prevenir las enfermedades profesionales de mayor prevalencia en Colombia. Dicho plan de trabajo fue incluido en el Plan Nacional de Salud Ocupacional y Seguridad en el trabajo 2013 – 2021, refrendando de esta manera el compromiso del Ministerio frente al tema de la prevención de las enfermedades profesionales.

Si bien los desórdenes músculo esqueléticos relacionados con el trabajo (DME) son entidades comunes y potencialmente discapacitantes, son prevenibles, por tal motivo, fue creada la Guía de Atención Integral de Salud Ocupacional (GATISO) basada en la evidencia para Desórdenes Músculo Esqueléticos (DME) relacionados con movimientos repetitivos de miembros superiores. A pesar de que la guía fue elaborada para prevenir, realizar el diagnóstico precoz y proporcionar tratamiento a los trabajadores en riesgo de sufrir o ser afectados por las enfermedades más comunes, el riesgo de sufrir una lesión de esta categoría sigue latente debido a que el periodo de tiempo sobre el cual se realiza la evaluación es muy reducido.

Si bien a durante los últimos años se han desarrollado herramientas para automatizar el proceso como lo es \textit{'ILENA'}\footnote{Herramienta computacional en donde por medio del dispositivo Kinect se determinan falencias en los movimientos a partir de las posiciones ergonómicas esperadas} propuesta por \parencite{Moya2015ModeloOcupacional}, se vuelve notable la inexactitud al momento de evaluar los miembros distales superiores\footnote{Manos, muñeca y antebrazo} ya que el sensor carece de precisión, de esta manera descuidando la zona donde se focalizan la mayoría de los desórdenes músculo esqueléticos padecidos, en particular, por personas que desempeñan labores de computo.

Por ende resultaría interesante contar la implementación un dispositivo que pueda ser utilizado en la oficina y permita monitorear y detectar movimientos ergonómicamente incorrectos realizados con los miembros distales superiores con el fin de incrementar el diagnóstico y prevenir las enfermedades profesionales de oficina de mayor predominio en Colombia y así fortalecer las herramientas disponibles para contrarrestar la problemática.

Partiendo de esta necesidad, se plantea el proyecto presente en busca de crear nuevas soluciones para la salud ocupacional, teniendo como objetivo diseñar y construir una solución que integre dispositivos de hardware existentes con software basado en inteligencia artificial que permita dar soporte a los procesos de monitoreo del puesto de trabajo, para la evaluación o estudio del mismo, lo anterior siendo fundamentado sobre tecnologías de desarrollo libre.