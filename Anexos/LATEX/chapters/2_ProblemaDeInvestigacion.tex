\chapter{PROBLEMA DE INVESTIGACIÓN}

\section{DESCRIPCIÓN DEL PROBLEMA}
Las instituciones de salud ocupacional son las encargadas de realizar los procesos de evaluación de puestos laborales, estos procesos permiten a las empresas contar con información de calidad para evitar trastornos ergonómicos que puedan perjudicar la salud de los trabajadores, sin embargo, estos procesos no son perfectos y se encuentran en constante optimización para incrementar los niveles de fiabilidad. Por tal motivo se busca fortalecer los procesos de salud ocupacional, pretendiendo garantizar que el entorno de trabajo se encuentre en armonía con las actividades que realiza el trabajador. %, lo anterior se refleja en la productividad, la calidad, la seguridad, la salud, la fiabilidad, la satisfacción con el trabajo y en el desarrollo personal de los trabajadores.\footcite[3]{MiguelAGonzalezPerez}

Actualmente se examinan las condiciones de los puestos de trabajo desde una perspectiva ergonómica con el propósito de obtener una detección temprana de condiciones o actos subestándar y finalmente culminar en los controles necesarios para disminuir la probabilidad de presentar alteraciones osteomusculares, desórdenes músculo esqueléticos o enfermedades laborales relacionadas. Sin embargo, estas revisiones solo se ejecutan en un intervalo de tiempo determinado durante el cual la persona que está siendo evaluada asume las posiciones correctas al sentirse observada e impide un diagnóstico efectivo de vicios posturales, además, las herramientas tecnológicas dispuestas para automatizar este proceso se encuentran diseñadas con sensores RGB-D que evalúan el cuerpo a gran escala, dejando de lado la precisión requerida para analizar los desórdenes músculo esqueléticos de la mano, muñeca y antebrazo

Lo descrito anteriormente, repercute en demoras para realizar los cambios en cuanto al mejoramiento en las condiciones y adaptación de los puestos de trabajo. Por tal motivo, se busca implementar una estrategia que permita tener un método de observación objetivo que perciba, identifique y puntualice los vicios posturales. Esto contribuiría a la exactitud y neutralidad al realizar una inspección de puesto laboral.

La empresa Latino BI Consulting presenta antecedentes de los síntomas descritos previamente en sus trabajadores, con lo cual, se convierte en una unidad de análisis sobre la cual implementar los métodos propuestos y analizar el proceso, el desarrollo y los resultados obtenidos que merecen atención en el futuro.

Por lo tanto se decide poner en marcha el proyecto Ergo-Sentinel (ErgoSent) en su primera versión, en donde se propone diseñar, desarrollar y probar un prototipo de software como soporte a la gestión de la revisión de puestos de trabajo para los miembros superiores al utilizar un mouse y teclado.

\section{FORMULACIÓN DEL PROBLEMA}
De acuerdo a la problemática descrita anteriormente, se plantea la siguiente pregunta que guiará el desarrollo de la propuesta de trabajo ¿Cómo desarrollar un prototipo de software de escritorio basado en inteligencia artificial que facilite el proceso de revisión de puestos de trabajo en una organización?